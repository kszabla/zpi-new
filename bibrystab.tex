 %Usuwa numeracje z naglowka. Zapewnia  dodanie do spisu tresci
\setcounter{secnumdepth}{-1}


%Gdy mamy dużą bibliografię to możemy wybierać pozycje,
%które cytujemy
%\nocite{ad-tg-80}

%Dodaje wszystkie pozycje z bibliografii
%\nocite{*}

%Po kazdym dodaniu nowej pozycji bibliograficznej
%z katalogu glownego uruchom: bibtex pracadyp
%\bibliographystyle{pdplain}
%\bibliography{tex/pracadyp}

\begin {thebibliography}{11}
\bibitem{Balcerzak2005} Balcerzak J., Pansiuk J.: \emph{Wprowadzenie do kartografii matematycznej}, Warszawa, OWPW~2005.
\bibitem{Barrett} Barrett R. i inni: \emph{Templates for the Solution of Linear Systems: Building Blocks for Iterative Methods1}, wersja elektorniczna Mathematics http://www.siam.org/books.
\bibitem{bjork} Bjork A., Dahlquist G.: \emph{Numerical Methods in Scientific Computing}, Philadelphia, SIAM~2002.
\bibitem{CCITTG4}CCITT, \emph{Facsimile Coding Schemes and Coding Control Functions for Group 4 Facsimile
Apparatus, Recommendation T.6, Volume VII, Fascicle VII.3, Terminal Equipment and
Protocols for Telematic Services, The International Telegraph and Telephone Consultative Committee (CCITT)}, Geneva, CCITT~1985.
\bibitem{drwal:mathematica2000} Drwal G, i in., \emph{Mathematica 4}, Gliwice, WPKJS~200.
\bibitem{Gdowski1982} Gdowski B.: \emph{Elementy geometrii rózniczkowej w zadaniach}, Warszawa, PWN~1982.
\bibitem{Gotlib2007} Gotlib D., Iwaniak A., Olszewski R.: \emph{GIS obszary zastosowań}, Warszawa, PWN~2007.
\bibitem{INTERGRAPHFileFormat1994} INTERGRAPH: \emph{INTERGRAPH RASTER FILE FORMAT REFERENCE GUIDE}, Alabama, Intergraph Corporation~1994.
\bibitem{Januszewski2006} Januszewski J.: \emph{Systemy satelitarne GPS, Galileo i inne}, Warszawa, PWN~2006.
\bibitem{kielbasinski1992}: Kiełbasiński A., Schwetlick H.: \emph{Numeryczna algebra liniowa}, Warszawa, WNT~1992.
\bibitem{Kincaid2006} Kincaid D.: \emph{Analiza numeryczna}, Warszawa, WNT~2006.
\bibitem{Lamparski2001}Lamparski J.: \emph{Navstar GPS od teorii do praktyki}, Olsztyn, WUW-M~2001.
\bibitem{Levine1994} Levine J.: \emph{Programowanie plików graficznych w C/C++}, New York, Wiley~1994.
\bibitem{Longley2006} Longley P. i inni: \emph{GIS teoria i praktyka}, Warszawa, PWN~2006.
\bibitem{GML:opengis} Open Geospatial Consortium Inc.: \emph{OpenGIS Geography Markup Language (GML) Encoding Standard, Version: 3.2.1},  OGC~2007.
\bibitem{GML:opengisimplemntation} Open Geospatial Consortium Inc.: \emph{OpenGIS® Geography Markup Language (GML) Implementation Specification}, OGC~2004.
\bibitem{Opera2002} Opera J.: \emph{Geometria róniczkowa i jej zastosowania}, Warszawa, PWN~2002.
\bibitem{Poczobut1982Geogeza} Odlanicki-Poczobut M.: \emph{Geodezja}, PPWK~1982.
\bibitem{Li2007}Li Y. i inni: \emph{GML Topology Data Storage Schema Design}, Chiba University~2007.
\bibitem{li2004GMLstorage}Li Y., Li J., Zhou S.: \emph{GML Storage}, A Spatial Database Approach,ER (Workshops), str 55-66, 2004.
\bibitem{Sayood2002} Sayood K.: \emph{Kompresja danych}, Warszawa, Rm~2002.
\bibitem{G52003} \emph{The Technical Instruction G-5, The Ground Cadastre and Buildings, The Main Surveying and
Cartographic Bureau}, Warszawa 2003.



\end {thebibliography}


\listoffigures

%\listoftables